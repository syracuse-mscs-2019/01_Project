\documentclass[oneside]{book}
\usepackage{lipsum}
\usepackage{listings}
\usepackage[legalpaper, portrait, margin=2in]{geometry}

\title{Project 1}
\date{\today}
\author{Michael Hrishenko, \LaTeX\ Journeyman}

\begin{document}
\maketitle
\textbf{Abstract}: Project one, purpose is mainly familiarization with HOL and Emacs, and also Latex reports. Multiple exercises and HOL and discussions concerning HOL errors, type constraints, and data structures \& mapping.
\newline
\\\textbf{Acknowledgements}: Professors Shiu-Kai Chin \& Susan Older
\tableofcontents

\chapter{Executive Summary}
All objectives complete. No problems to report other than lack of time due to operations tempo with Iranian tensions. Things have calmed and we have a weekend again, which I will use to catch up on work. Of note I am currently using Emacs run through the XServer on my Windows, while Emacs runs on Windows Subsystem for Linux (WSL) to avoid virtualization headaches. The setup is like a dream compared to the VM, which I realize is the catch-all solution but for those of us running Microsoft I believe this solution might be preferred. First time using Emacs so still some familiarization on that front. Otherwise straightforward and fully complete.

\chapter{Exercise 2.5.1}
\section{Problem Statement}
\begin{enumerate}
    \item Start up Emacs with a fresh file ex-2-5-1.sml.
    \item Start HOL inside of Emacs, highlight the definition of timesPlus, and send the region to HOL.
    \item Evaluate the expression timesPlus 100 27 within HOL. If you’ve done things correctly, you should get a pair of integers as a result. Note: when you start HOL within Emacs, a second window opens below or on the right of your source code. This is the *HOL* buffer. Move your cursor to this buffer by using your mouse or by typing C-x o, which moves the cursor among the various Emacs buffers/windows.
    \item Kill the HOL process while preserving the *HOL* window by moving your cursor to the *HOL* window and typing C-D. Save the contents of the *HOL* window under the name ex-2-5-1.trans.
\end{enumerate}
\section{Code}
    \begin{lstlisting}
(* Name: Michael Hrishenko *)
(* Email: mahrishe@syr.edu *)
fun timesPlus x y = (x*y, x+y);
    \end{lstlisting}
\section{Test Cases}
\begin{enumerate}
    \item Input:
        \begin{lstlisting}
timesPlus 100 27;
        \end{lstlisting}
    \item Output:
        \begin{lstlisting}
val it = (2700, 127): int * int
        \end{lstlisting}
\end{enumerate}

\chapter{Exercise 3.4.1}
\section{Problem Statement}
Create a file ex-3-4-1.sml as your sourcefile. Define the following values in ML. Please include comments similar to those in the examples we have shown in this Chapter. Execute your final source code in the HOL interpreter and create a transcript file ex-3-4-1.trans by saving the *HOL* window in Emacs to ex-3-4-1.trans. 
\begin{enumerate}
    \item Devise the list of pairs [(0,"Alice"), (1,"Bob"), (3,"Carol"),(4,"Dan")] and assign it the name listA.
    \item Using listA and pattern matching, create the following value assignments: elB has the value (0,"Alice") and listB has the value [(1,"Bob"),(3,"Carol"),(4,"Dan")]
    \item Using elB, listB, and pattern matching, create the following value assignments: elC1 has the value 0, elC2 has the value ”Alice”, elC3 has the value (1, "Bob"), elC4 has the value (3, "Carol"), and elC5 has the value (4, "Dan").
\end{enumerate}
\section{Code}
    \begin{lstlisting}
(********************************************************)
(* Exercise 3.4.1 *)
(* Author: Michael Hrishenko *)
(* Date: 17JAN2020 *)
(********************************************************)
val listA = [(0,"Alice"), (1,"Bob"), (3,"Carol"),(4,"Dan")];
val (elB :: listB) = listA;
val (elC1, elC2) = elB;
val (elC3 :: elC4 :: elC5 :: []) = listB;
    \end{lstlisting}
\section{Test Cases}
\begin{enumerate}
    \item Input:
        \begin{lstlisting}
val (elB :: listB) = listA;
        \end{lstlisting}
    \item Output:
        \begin{lstlisting}
val elB = (0, "Alice"): int * string
val listB = [(1, "Bob"), (3, "Carol"), (4, "Dan")]: (int * string) list
        \end{lstlisting}
    \item Input:
        \begin{lstlisting}
val (elC1, elC2) = elB;
        \end{lstlisting}
    \item Output:
        \begin{lstlisting}
val elC1 = 0: int
val elC2 = "Alice": string
        \end{lstlisting}
    \item Input:
        \begin{lstlisting}
val (elC3 :: elC4 :: elC5 :: []) = listB;
        \end{lstlisting}
    \item Output:
        \begin{lstlisting}
val elC3 = (1, "Bob"): int * string
val elC4 = (3, "Carol"): int * string
val elC5 = (4, "Dan"): int * string
        \end{lstlisting}
\end{enumerate}

\chapter{Exercise 3.4.2}
\section{Problem Statement}
Create a file ex-3-4-2.sml as your sourcefile. Define the following values in ML. Please
include comments similar to those in the examples we have shown in this Chapter. Execute your final source
code in the HOL interpreter and create a transcript file ex-3-4-2.trans by saving the *HOL* window in
Emacs to ex-3-4-2.trans.
\begin{enumerate}
    \item Evaluate each of the assignments in the order in which they appear in HOL. Store the results in your ex-3-4-2.trans file.
    \item Explain in your own words what the errors are that HOL detects. Include your answers as comments in your source code.
\end{enumerate}
\section{Code}
    \begin{lstlisting}
(********************************************************)
(* Exercise 3.4.2 *)
(* Author: Michael Hrishenko *)
(* Date: 17JAN2020 *)
(********************************************************)
val (x1,x2,x3) = (1,true,"Alice");
val pair1 = (x1,x3);
val list1 = [0,x1,2];
val list2 = [x2,x1];
val list3 = (1 :: [x3]);
    \end{lstlisting}
\section{Test Cases}
\begin{enumerate}
    \item val (x1,x2,x3) = (1,true,"Alice");
    \begin{itemize}
        \item NO ERRORS
    \end{itemize}
    \item val pair1 = (x1,x3);
    \begin{itemize}
        \item NO ERRORS
    \end{itemize}
    \item val list1 = [0,x1,2];
    \begin{itemize}
        \item NO ERRORS
    \end{itemize}
    \item val list2 = [x2,x1];
    \begin{itemize}
        \item Elements in a list have different types: Since x2 is "true" and x1 is '1', the list cannot be constructed. This is a HOL feature, whereas Python lists and other languages may hold elements of different types, HOL is designed to root out type errors and prove assurance in design, so mixed types are not allowed.
    \end{itemize}
    \item val list3 = (1 :: [x3]);
    \begin{itemize}
        \item NO ERRORS
    \end{itemize}
\end{enumerate}

\chapter{Appendices}
\appendix
\section{Appendix A}

\lstinputlisting{ex-2-5-1.sml}

\section{Appendix B}

\lstinputlisting{ex-3-4-1.sml}

\section{Appendix C}

\lstinputlisting{ex-3-4-2.sml}

\end{document}